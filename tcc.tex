\documentclass[12pt]{article}

\usepackage{sbc-template} 

\usepackage{graphicx,url}

\usepackage[brazil]{babel}

\usepackage{easyReview}

\sloppy

\title{Pagamentos Automatizados Programáveis em Carteiras Auto-Custodiadas: Uma Exploração Técnica}

\author{Ana Julia Bittencourt Fogaça\inst{1}, Saulo Popov Zambiasi\inst{2} }

\address{Universidade do Sul de Santa Catarina (UNISUL)\\
	Tubarão - SC - Brasil
	\nextinstitute
	Universidade do Sul de Santa Catarina (UNISUL)\\
	Florianópolis - SC - Brasil
	\email anajuliabit@gmail.com, saulopz@gmail.com
}

\begin{document}

\maketitle

\begin{abstract}
\end{abstract}

\section{Introdução}
A tecnologia blockchain, primeiramente introduzida por Satoshi Nakamoto em 2008, é identificada
como uma megatendência computacional capaz de revolucionar múltiplos setores industriais\cite{1}.
As características distintas de segurança, transparência e rastreabilidade inerentes à blockchain
têm incentivado uma ampla gama de setores a explorar seu uso na reestruturação de suas operações
fundamentais. A aplicabilidade dessa tecnologia ultrapassa o domínio das criptomoedas, abarcando
setores como pagamentos, gerenciamento de identidade, saúde, eleições governamentais e
outros\cite{2}.

A publicação do whitepaper do Ethereum em 2014 marcou um progresso notável no desenvolvimento da
tecnologia blockchain\cite{3}. Ao contrário do Bitcoin, que foi inicialmente projetado como um
sistema de pagamento digital, o Ethereum introduziu uma funcionalidade revolucionária na tecnologia
blockchain: os contratos inteligentes. A principal inovação proporcionada pelo Ethereum é a
integração de uma máquina virtual capaz de executar códigos em linguagens de programação
\textit{Turing complete} na blockchain, possibilitando a criação de aplicativos descentralizados.
Esses aplicativos substituem o sistema de back-end por contratos inteligentes que operam em uma
blockchain\cite{7}. No entanto, apesar de seu potencial imenso, a complexidade associada à
aplicação prática desta tecnologia é um dos obstáculos para sua adoção em larga escala\cite{11}.
Para utilizar os aplicativos descentralizados (DApps) de maneira segura, os usuários precisam ter
conhecimento técnico em criptografia para manter suas chaves privadas seguras\cite{6}, devido à
natureza das carteiras auto-custodiadas na blockchain, entraremos em mais detalhes a seguir. Esta
não é a realidade para a maioria dos usuários da internet, o que dificulta a adoção de DApps. Além
disso, a experiência do usuário final é insatisfatória quando comparada à maneira como usamos a
internet hoje. É como se fosse necessário inserir sua senha (chave privada) para cada ação que você
realiza que não seja apenas consumir dados - em termos técnicos, uma operação de leitura.

Em resposta ao desafio emergente, propostas inovadoras são regularmente introduzidas com a intenção
de aprimorar a experiência do usuário e, por extensão, promover a adoção de DApps. Uma dessas
propostas recentes é a \textit{Account Abstraction} (AA). AA apresenta uma inovação significativa,
proporcionando maior flexibilidade no processo de validação de transações e permitindo autorizações
personalizadas que não dependem necessariamente de uma assinatura criptográfica. Esta
funcionalidade tem o potencial de aprimorar consideravelmente a experiência do usuário no
ecossistema Ethereum\cite{12}, abrindo caminho para novos casos de uso que anteriormente eram
inviáveis dentro de uma blockchain. Um desses casos de uso foi recentemente o foco de investigação
da Visa, uma empresa líder em soluções de pagamento. A Visa sugeriu a possibilidade de realizar
pagamentos automáticos na blockchain sem a necessidade de exposição de chaves privadas,
aproveitando os benefícios da AA\cite{9}. Contudo, os detalhes técnicos ou o código-fonte para tal
implementação não foram divulgados. Esta pesquisa se propõe a explorar profundamente essa
funcionalidade e propor uma implementação prática para viabilizar pagamentos automáticos em
carteiras auto-custodiadas. O objetivo é fornecer uma perspectiva esclarecedora e diretrizes
pragmáticas que possam servir como referência para futuras aplicações de pagamentos automatizados
em blockchains como o Ethereum.

A estrutura deste artigo é organizada da seguinte maneira: Seguindo esta introdução, avançamos para
uma revisão bibliográfica abrangente, iniciando com uma explanação aprofundada do funcionamento do
Ethereum, esboçando as responsabilidades da Ethereum Virtual Machine (EVM), elucidando os
diferentes tipos de contas e esclarecendo as propostas da AA. Adicionalmente, empreendemos uma
análise do estudo conduzido pela Visa. Na seção de desenvolvimento, propomos uma implementação em
Solidity com o intuito de habilitar pagamentos automáticos programáveis em carteiras
auto-custodiadas. A nossa discussão culmina com uma reflexão sobre as implicações e possíveis
aplicações desta nova funcionalidade nas blockchains.

\section{Revisão Bibliográfica}\label{sec:revisao}

Como mencionado anteriormente, a característica distintiva do Ethereum reside em sua capacidade de
executar códigos, mais especificamente, contratos inteligentes, em uma máquina virtual denominada
\textit{Ethereum Virtual Machine} (EVM). A EVM, uma máquina virtual global que opera em um formato
de instância única, e é executada repetidamente em uma variedade de computadores em todo o mundo.
Cada nó no Ethereum mantem uma cópia local da EVM, que é responsável por validar e executar
contratos inteligentes. Qualquer mudança de estado decorrente da execução desses contratos
inteligentes é registrada no Ethereum\cite{6}.O Ethereum tem como objetivo permitir a execução de
aplicações e scripts arbitrários que operam em transações, utilizando uma blockchain para
sincronizar o estado global de maneira totalmente verificável por qualquer participante do
sistema\cite{13}. O estado, que é único e compartilhado entre todos os nós, confere uma
característica essencial aos contratos inteligentes: a necessidade de serem determinísticos. A
linguagem mais popular para a criação de contratos inteligentes é o Solidity\cite{16}. O Solidity é
uma linguagem de programação de alto nivel e precisa ser compilada para código EVM - bytecode que a
EVM pode executar nativamente\cite{15}. Devido a essas características, o Ethereum é frequentemente
descrito como um 'computador mundial de propósito geral'\cite{6}.

A compreensão clara das diferenças entre os dois tipos de contas no Ethereum é fundamental para um
entendimento completo do funcionamento do Ethereum. No Ethereum, existem duas categorias distintas
de contas: as contas de propriedade externa (\textit{externally owned accounts} - EOAs), que são
controladas por chaves privadas, e as contas de contrato (\textit{contract accounts} - CAs), que
são controladas pela lógica incorporada no código dos contratos inteligentes associados a
elas\cite{3}. Em termos técnicos, as CAs têm a propriedade 'codeHash' - o hash do código do
contrato inteligente na linguagem da EVM, que é o código executado sempre que uma CA é o
destinatário de uma transação, e o campo 'storageRoot' - que representa o estado do contrato
inteligente preenchidos. Em contraste, as EOAs têm ambos os campos vazios\cite{15}.

O fato de as Externally Owned Accounts (EOAs) serem controladas unicamente por uma chave privada
que emprega o esquema de assinatura ECDSA\cite{17} para autenticar transações restringe a utilidade
da blockchain. Isso exige que os usuários possuam conhecimento técnico em armazenamento de chaves
privadas e na tarefa de assinar criptograficamente cada ação que desejam que resulte na alteração
do estado global da blockchain. Esta tarefa não é trivial para muitos, evidenciado pela
popularidade de carteiras com maior grau de segurança, como as carteiras multisig e as carteiras
físicas como a Ledger Nano S\cite{19} e a Trezor\cite{20} para gerenciar contas de blockchains.
Outro ponto crucial a destacar é que o algoritmo ECDSA pode se tornar obsoleto com a computação
quântica\cite{18}, o que sublinha a urgência de buscar alternativas.

Como o Ethereum é uma rede descentralizada e de código aberto, a comunidade constantemente formula
propostas de melhoria conhecidas como Ethereum Improvement Proposals (EIPs). A EIP-4337\cite{5},
também denominada Account Abstraction (AA), introduz uma inovação radical no modelo convencional de
contas no Ethereum, sugerindo a utilização de contratos inteligentes em vez das tradicionais EOAs.
Esta proposta tem o potencial de abrir novos casos de uso na plataforma.

Ao proporcionar maior flexibilidade para as contas de usuários, AA desencadeia uma série de novas
capacidades, destacando-se a possibilidade de autorização personalizada sem a obrigatoriedade do
uso de ECDSA, assim como nas EOAs. Isso permite a adaptação da lógica de autorização de acordo com
necessidades específicas. Através da EIP-4337, podemos estabelecer regras de validação que não
dependem necessariamente da assinatura do proprietário da conta, o que representa uma inovação para
as blockchains. Até então, a autorização de transações era exclusivamente baseada na atomicidade
das assinaturas criptográficas, exigindo que a aprovação ocorresse instantaneamente, sem a
possibilidade de pré-aprovação.Considerando que a EIP-4337 é uma proposta recente, concebida no
final de 2021 e oficialmente implementada em março de 2023, há uma lacuna notável em estudos
acadêmicos que aprofundem seu funcionamento. Portanto, é imperativo que nos aprofundemos nos
detalhes técnicos da AA para compreender como ela pode ser empregada para viabilizar pagamentos
automáticos programáveis em carteiras autogerenciadas.

\bibliographystyle{sbc}
\bibliography{tcc}

\end{document}

