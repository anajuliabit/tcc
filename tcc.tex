\documentclass[12pt]{article}

\usepackage{sbc-template}

\usepackage{graphicx,url}

\usepackage[brazil]{babel}

\usepackage{easyReview}

\sloppy

\title{Pagamentos Automatizados Programáveis em Carteiras Auto-Custodiadas: Uma Exploração Técnica}

\author{Ana Julia Bittencourt Fogaça\inst{1}, Saulo Popov Zambiasi\inst{2} }

\address{Universidade do Sul de Santa Catarina (UNISUL)\\
	Tubarão - SC - Brasil
	\nextinstitute
	Universidade do Sul de Santa Catarina (UNISUL)\\
	Florianópolis - SC - Brasil
	\email anajuliabit@gmail.com, saulopz@gmail.com
}

\begin{document}

\maketitle

\begin{abstract}
\end{abstract}

\begin{resumo}
  À medida que a tecnologia blockchain continua a ganhar adoção, percebe-se um crescimento na demanda por
  funcionalidades que viabilizem sua aplicação em situações reais. No âmbito deste artigo técnico, nos
  inspiramos na recente publicação da Visa como premissa para a exploração da implementação de pagamentos
  programáveis automatizados em carteiras de auto custódia. Adentramos na esfera do conceito de Abstração de
  Contas e propomos uma execução codificada em Solidity com o propósito de habilitar pagamentos recorrentes
  originados diretamente de carteiras sob controle do usuário. Por meio desta sondagem, nosso intuito
  primordial é fornecer perspectivas elucidativas e orientações pragmáticas relativas à implementação de
  pagamentos automáticos no contexto das finanças descentralizadas.
\end{resumo}

\section{Introdução}

Primeiramente introduzida por Satoshi Nakamoto em 2008, a inovação da tecnologia blockchain tem
sido reconhecida como uma das "megatendências" computacionais com capacidade para reestruturar a
paisagem global nas próximas décadas (World Economic Forum, 2015). A singularidade inerente à
tecnologia blockchain, englobando aspectos de segurança, transparência e rastreabilidade, tem
estimulado uma gama diversificada de setores industriais a explorarem sua utilização na
reformulação de suas operações centrais. A amplitude das suas aplicações estende-se para além do
domínio da criptomoeda, permeando domínios como a gestão de identidade, saúde, eleições
governamentais, entre outros \cite{PuLam2021}.

Com a publicação do \textit{whitepaper} do Ethereum em 2014\cite{Buterin2014}, um marco
significativo foi estabelecido no desenvolvimento da tecnologia blockchain. Em contraste com o
Bitcoin, que foi inicialmente concebido como um sistema de pagamento digital, o Ethereum introduziu
uma nova faceta à tecnologia blockchain: os Smart Contracts. Essa inovação foi projetada
originalmente para se beneficiar dos atributos da blockchain, implementando automaticamente os
termos acordados entre duas partes ao formalizar um contrato num ambiente desprovido de confiança,
uma característica que conduziu à denominação de \textit{Smart Contract} para esse código de
software\cite{Pinna2019}.

A clara compreensão dos detalhes operacionais das contas no ecossistema Ethereum torna-se
imperativa para um entendimento completo de como funcionam os Smart Contracts. No Ethereum, existem
essencialmente duas categorias distintas de contas: as \textit{externally owned accounts} (EOAs) e
as \textit{contract accounts}.

As EOAs, ou contas de propriedade externa, podem ser estabelecidas através de carteiras como a
MetaMask. Essas contas detêm uma chave privada que autoriza o controle sobre o acesso aos fundos ou
contratos associados. Em contraste, as \textit{contract accounts}, ou contas de contrato,
apresentam características diferenciadas das EOAs. Uma conta de contrato abriga o código de um
\textit{smart contract}, uma funcionalidade ausente numa simples EOA. Além disso, uma conta de
contrato não dispõe de uma chave privada. Ao invés disso, é detida (e controlada) pela lógica
incorporada no código do seu \textit{smart contract} - um programa de software gravado na
blockchain Ethereum no momento de criação da conta de contrato e operado pela Ethereum Virtual
Machine (EVM) \cite{Antonopoulos2018}.

As contas de contrato possuem endereços, à semelhança das EOAs, e também são capazes de enviar e
receber Ether. Contudo, quando uma transação é destinada a um endereço de contrato, ela provoca a
execução desse contrato na EVM, usando a transação e os dados da transação como entrada. Além do
Ether, as transações podem conter dados que especificam qual função particular do contrato deve ser
ativada e quais parâmetros devem ser fornecidos a essa função. Assim, as transações podem invocar
funções dentro dos contratos \cite{Antonopoulos2018}.

Agora que elucidamos a distinção entre as contas de propriedade externa (EOAs) e as contas de
contrato, podemos aprofundar nossa compreensão no contexto das aplicações descentralizadas (DApps).
Essas aplicações substituem a infraestrutura tradicional de back-end por smart contracts operando
em blockchains como o Ethereum. No entanto, a adoção desses DApps enfrenta o desafio premente de
proporcionar uma experiência de usuário comparável àquela oferecida pelas aplicações centralizadas.

Para enfrentar esses desafios intrínsecos, surgem esforços incessantes visando aprimorar e
simplificar a experiência do usuário no ambiente do Ethereum. Como o Ethereum constitui uma rede
descentralizada e de código aberto, a comunidade formula continuamente propostas de melhoria
conhecidas como \textit{Ethereum Improvement Proposals} (EIPs). Uma proposta relevante para nossa
discussão é a EIP-4337.\cite{Buterin2021}

A EIP-4337, popularmente denominada como \textit{Account Abstraction} (AA), propõe uma inovação
radical no modelo convencional de contas do Ethereum, sugerindo a utilização de \textit{smart
  contracts} no lugar das tradicionais EOAs. Essa proposta tem o potencial de desencadear a
exploração de novos casos de uso na plataforma.

Na essência, a \textit{Account Abstraction} busca consolidar contas de usuários e contratos
inteligentes em um único tipo de conta Ethereum. Assim, as contas de usuários passariam a operar
como contratos inteligentes. Este conceito, apesar de simples, traz implicações significativas: ele
confronta e busca modificar os requisitos inflexíveis inerentes às transações Ethereum atuais, que
são rigidamente codificados no protocolo Ethereum, tais como a necessidade de uma assinatura ECDSA
válida, um nonce válido e saldo suficiente para cobrir os custos de computação \cite{Visa2023}.

Ao oferecer maior flexibilidade no processo de validação de transações, a Account Abstraction (AA)
desencadeia uma série de novas capacidades. Entre alguns casos de uso, destacam-se a possibilidade
de autorização personalizada, adaptando a lógica de autorização de acordo com as necessidades e
permitindo a escolha de esquemas criptográficos alternativos em caso de descoberta de
vulnerabilidades no ECDSA, garantindo a segurança contínua (1). Além disso, a AA permite a
abstração do custo do gás, possibilitando que transações iniciadas por contas de contrato sejam
pagas pelo proprietário ou por um terceiro chamado "paymaster", que arca com as taxas de gás da
transação (Buterin, 2021). Essa abordagem também facilita a entrada de usuários inexperientes, pois
o paymaster pode ser financiado com moeda fiduciária fora da cadeia, eliminando a necessidade de
lidar diretamente com ether. A AA ainda possibilita a implementação de carteiras com recuperação
social, onde um contrato de conta pode ser programado com um endereço de backup controlado por uma
parte confiável ou usando um esquema de múltiplas assinaturas, permitindo a recuperação da carteira
em casos de perda de chaves privadas ou frases de backup. O caso de uso mais importante para o
contexto deste artigo é a capacidade de realizar pagamentos automáticos, adicionando regras de
validação que não dependem necessariamente de assinaturas (Visa, 2023). Isso é inédito para
blockchains, uma vez que pagamentos automáticos são impossíveis em EOAs, pois assinaturas são
necessárias para todas as transações.

Diante desse cenário emergente, a utilização da blockchain na esfera dos pagamentos programáveis
automatizados desponta como foco central desta pesquisa. A Visa, reconhecida por suas soluções de
pagamento, está empenhada em explorar métodos que simplifiquem o processo, permitindo aos usuários
realizar auto-pagamentos na blockchain sem a necessidade de divulgar suas chaves privadas.

Neste trabalho, iniciaremos com a análise meticulosa da publicação recente da Visa sobre o assunto.
Em seguida, aprofundaremos nosso entendimento do conceito de Abstração de Contas e, finalmente,
sugeriremos uma implementação em Solidity que habilite a execução de pagamentos automáticos
programáveis em carteiras auto-custodiadas. Com a implementação proposta, exploraremos as
implicações e oportunidades que se apresentam para o emergente campo de pagamentos.

\section{Revisão Bibliográfica}\label{sec:revisao}

\subsection{A publicação da Visa}
Em sua recente contribuição para a expansão do campo de pagamentos automáticos programáveis em
carteiras auto-custodiadas via blockchain, a Visa apresentou revelações significativas numa
publicação técnica intitulada "Autopayments via Account Abstraction". As próximas linhas fornecem
uma síntese das conclusões primárias e propostas emergentes desta publicação.

Na sua abordagem, a Visa introduz um mecanismo inovador que simplifica a possibilidade do usuário
de executar autopagamentos, prescindindo do uso da chave privada associada à sua identidade. O
intuito subjacente é viabilizar autopagamentos para comerciantes sem a necessidade de expor a chave
privada do usuário a qualquer servidor de terceiros. Como alternativa, um contrato inteligente é
capacitado para processar um autopagamento em nome do usuário para o comerciante destinatário, sem
a exigência da chave privada do usuário.

O modelo concebido pela Visa concede ao contrato inteligente a autoridade de conduzir pagamentos
automáticos aos comerciantes associados ao usuário, estando sempre condicionado à ratificação deste
último. Esta ratificação pode ser obtida através do fornecimento de dados que o contrato
inteligente está autorizado a utilizar para realizar os autopagamentos em representação do usuário.
Em essência, o usuário pode pré-autorizar a transação, habilitando o contrato inteligente a
processar o pagamento em seu nome quando uma solicitação é encaminhada pelo comerciante. A Visa
sugere ainda que o usuário possa compilar uma lista de permissões, onde poderá pré-autorizar
transações com pagadores predeterminados, como comerciantes, outros usuários, entre outros.

Na monografia técnica mencionada, a Visa recorreu a um conceito recente e a uma das principais
propostas de desenvolvedores do Ethereum conhecida como Abstração de Conta para investigar a
implementação de contratos inteligentes que viabilizem pagamentos programáveis automáticos.
Propuseram uma inovadora solução para uma aplicação real de pagamentos automáticos, demonstrando
como estruturar um contrato inteligente para uma carteira autogerida capaz de retirar fundos
automaticamente, sem necessitar da participação ativa do usuário em cada instante para instruir e
realizar pagamentos numa blockchain.

Na seção Desenvolvimento deste artigo, exploraremos a proposta da Visa em maior detalhe, bem como
as implicações e oportunidades que ela apresenta para o campo emergente de pagamentos. Iremos
propor uma implementação escrita em Solidity, a linguagem de programação mais utilizada na
plataforma Ethereum, que permitirá a execução de pagamentos automáticos programáveis em carteiras
auto-custodiadas. Apesar da visa ter descrito a arquitetura de uma conta de carteira autogerida
capaz de realizar pagamentos automáticos, não foi fornecida uma implementação de referência. A
implementação proposta neste artigo visa preencher essa lacuna. Mas antes de entrarmos em detalhes
sobre a implementação, precisamos entender o que é a Abstração de Contas É disso que trataremos na
próxima seção.

\subsection{Abstração de Conta}

A Abstração de Conta é um conceito que vem sendo explorado para aumentar a flexibilidade e
funcionalidade das contas na rede Ethereum.

Conforme destacado por Andreas M. Antonopoulos em "Mastering Ethereum" \cite{Antonopoulos2018},
duas categorias primárias de contas são estabelecidas na rede Ethereum: \textit{Externally Owned
  Accounts} (EOAs) e \textit{Contract Accounts}.

As EOAs, as contas de propriedade externa, podem ser instituídas através de interfaces de carteira
como a MetaMask. Estas contas estão vinculadas a uma chave privada que proporciona controle sobre o
acesso a fundos e contratos correspondentes. Em contrapartida, as \textit{Contract Accounts}, ou
contas de contrato, diferem fundamentalmente das EOAs. Uma conta de contrato hospeda o código de um
\textit{smart contract}, recurso ausente em uma EOA convencional. Além disso, uma conta de contrato
não possui uma chave privada. Em vez disso, a posse (e controle) está estruturada em torno da
lógica do código de seu \textit{smart contract}, um programa de software que é registrado na
blockchain Ethereum durante a criação da conta de contrato e operado pela Ethereum Virtual Machine
(EVM) \cite{Antonopoulos2018}.

As contas de contrato, assim como as EOAs, possuem endereços e têm a capacidade de enviar e receber
Ether. No entanto, quando uma transação é direcionada a um endereço de contrato, isso aciona a
execução desse contrato na EVM, usando a transação e os dados da transação como entrada. Além do
Ether, as transações podem incorporar dados que indicam qual função específica do contrato deve ser
ativada e quais parâmetros devem ser fornecidos para essa função. Desta forma, as transações têm a
capacidade de invocar funções dentro dos contratos \cite{Antonopoulos2018}.

A Abstração de Conta sugere um desenvolvimento transformador na maneira como essas contas
funcionam, propondo que todas as contas na rede Ethereum tenham a potencialidade de operar como
Contas de Contrato. Isso implica que cada conta poderia abrigar sua própria lógica de operação,
conduzindo a um grau de personalização e funcionalidade sem precedentes. Por exemplo, uma conta
poderia ser programada para gerenciar transações de uma maneira específica ou para se defender
contra certos tipos de ataques.

\section{Desenvolvimento}

\subsection{O problema}

Ilustração detalhada de um cenário hipotético que demonstra a necessidade de pagamentos
automatizados em uma carteira auto-custodiada. Análise das limitações enfrentadas pelos usuários ao
tentar agendar pagamentos automáticos em uma blockchain como o Ethereum.

\subsection{Abordagem da solução}

Visão geral da solução proposta, aproveitando a Abstração de Contas e contratos inteligentes.
Introdução de contas delegáveis e do conceito de regras de validade programáveis para transações.
Demonstração de como os pagamentos automáticos podem ser implementados usando contratos
inteligentes de pagamento automático pré-aprovados.

\subsection{Implementação técnica}

Explicação dos passos técnicos necessários para configurar e configurar contas delegáveis em
carteiras auto-custodiadas. Visão sobre o fluxo de transação e interação entre o contrato
inteligente de pagamento automático e as carteiras controladas pelo usuário.

\subsection{Vantagens e Potenciais Aplicações}:
Discussão dos benefícios e vantagens oferecidos pela solução proposta para pagamentos automáticos.
Exploração de casos de uso potenciais além de pagamentos recorrentes, como serviços de recuperação de conta de terceiros e gestão de ativos.

\section{Conclusão}\label{sec:conclusao}

Reflexão sobre a importância dos pagamentos programáveis e o potencial para inovações futuras no
espaço blockchain. Pesquisa Futura e Considerações Identificação de possíveis áreas para futuras
pesquisas e desenvolvimento na automatização de pagamentos em carteiras de autoguarda. Discussão
sobre possíveis desafios e considerações a serem tratados na implementação de pagamentos
automáticos em escala.

\bibliographystyle{sbc}
\bibliography{sbc-template}

\end{document}

