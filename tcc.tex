\documentclass[12pt]{article}

\usepackage{sbc-template} 

\usepackage{graphicx,url}

\usepackage[brazil]{babel}

\usepackage{easyReview}

\sloppy

\title{Pagamentos Automatizados Programáveis em Carteiras Auto-Custodiadas: Uma Exploração Técnica}

\author{Ana Julia Bittencourt Fogaça\inst{1}, Saulo Popov Zambiasi\inst{2} }

\address{Universidade do Sul de Santa Catarina (UNISUL)\\
	Tubarão - SC - Brasil
	\nextinstitute
	Universidade do Sul de Santa Catarina (UNISUL)\\
	Florianópolis - SC - Brasil
	\email anajuliabit@gmail.com, saulopz@gmail.com
}

\begin{document}

\maketitle

\begin{abstract}
\end{abstract}

\begin{resumo}
  (provisório) À medida que a tecnologia blockchain continua a ganhar adoção, percebe-se um crescimento na demanda por funcionalidades que viabilizem sua aplicação em situações reais. No âmbito deste artigo técnico, nos inspiramos na recente publicação da Visa como premissa para a exploração da implementação de pagamentos
  programáveis automatizados em carteiras de auto custódia. Adentramos na esfera do conceito de Abstração de
  Contas e propomos uma execução codificada em Solidity com o propósito de habilitar pagamentos recorrentes
  originados diretamente de carteiras sob controle do usuário. Por meio desta sondagem, nosso intuito   primordial é fornecer perspectivas elucidativas e orientações pragmáticas relativas à implementação de
  pagamentos automáticos no contexto das finanças descentralizadas.
\end{resumo}
\comment{RESUMO}{Sim, nem precisa fazer nada aqui por enquanto. Só vamos ver mais pro final.}

\section{Introdução}
A tecnologia blockchain, primeiramente introduzida por Satoshi Nakamoto em 2008, é identificada
como uma megatendência computacional capaz de revolucionar múltiplos setores industriais\cite{1}.
As características distintas de segurança, transparência e rastreabilidade inerentes à blockchain
têm incentivado uma ampla gama de setores a explorar seu uso na reestruturação de suas operações
fundamentais. A aplicabilidade dessa tecnologia ultrapassa o domínio das criptomoedas, abarcando
setores como pagamentos, gerenciamento de identidade, saúde, eleições governamentais e
outros\cite{2}.

Com a divulgação do \textit{whitepaper} Ethereum em 2014, um marco significativo foi estabelecido
no desenvolvimento da tecnologia blockchain\cite{3}. Distinto do Bitcoin, que foi originalmente
idealizado como um sistema de pagamento digital, Ethereum introduziu uma funcionalidade inovadora
na tecnologia blockchain: os contratos inteligentes (\textit{smart contracts}). Esta inovação foi
concebida para se valer dos atributos da blockchain, implementando automaticamente os termos
acordados entre duas partes ao formalizar um contrato em um ambiente desprovido de confiança, uma
característica que levou à denominação de \textit{smart contract} para esse código de
software\cite{4}. No entanto, apesar do seu grande potencial, a complexidade associada à aplicação
prática da tecnologia é um dos obstáculos para a sua adoção em larga escala\cite{11}. Para utilizar
de forma segura os aplicativos descentralizados (DApps) - aplicativos que substituem o sistema de
back-end por contratos inteligentes, os usuários precisam possuir conhecimentos técnicos em
criptografia para manter sua chave privada segura\cite{6}.

Em face deste desafio, o presente estudo pretende explorar tecnicamente a ERC-4337\cite{5}, também
conhecida como \textit{Account Abstraction (AA)}, para a execução de pagamentos automáticos em
carteiras auto-custodiadas. A ERC-4337 apresenta uma inovação substancial, proporcionando maior
flexibilidade no processo de validação de transações e permitindo autorizações personalizadas, que
não necessitam necessariamente de uma assinatura criptográfica. Esta capacidade tem o potencial de
aprimorar consideravelmente a experiência do usuário no ecossistema Ethereum\cite{12}. Esta
pesquisa procurará entender este recurso em profundidade e propor uma implementação prática para
habilitar pagamentos automáticos em carteiras auto-custodiadas. O objetivo é oferecer uma visão
esclarecedora e orientações pragmáticas relacionadas à implementação de pagamentos que possam
servir como referência para futuras aplicações. A pesquisa foi inspirada por um estudo conduzido
pela Visa, empresa líder em soluções de pagamento. A Visa propôs a ideia de realizar pagamentos
automáticos na blockchain sem a necessidade de revelar chaves privadas utilizando os benefícios da
ERC-4337\cite{9}. No entanto, os detalhes técnicos ou o código-fonte para tal implementação não
foram disponibilizados.

A estrutura deste artigo é a seguinte: após esta introdução, procederemos com uma revisão
bibliográfica detalhada, começando por um overview do Ethereum, abordando elementos da linguagem de
programação Solidity, da Ethereum Virtual Machine (EVM) e da ERC-4337. Na seção de desenvolvimento,
analisaremos o estudo da Visa e proporemos uma implementação em Solidity para habilitar pagamentos
automáticos programáveis em carteiras autogerenciadas. Concluiremos com uma discussão sobre as
implicações e possíveis aplicações desta nova funcionalidade nas blockchains, revelando
oportunidades inovadoras no campo dos pagamentos.

\section{Revisão Bibliográfica}\label{sec:revisao}

\subsection{Ethereum}
A compreensão clara das diferenças entre os dois tipos de contas no Ethereum é fundamental para um
entendimento completo do funcionamento dos contrato inteligentes. No Ethereum, existem
essencialmente duas categorias distintas de contas: as contas de propriedade externa
(\textit{externally owned accounts} - EOAs) e as contas de contrato (\textit{contract accounts}).
As EOAs, podem ser criadas e controladas por meio de carteiras como a MetaMask. Essas contas
possuem uma chave privada que concede controle sobre o acesso aos fundos ou contratos associados.
Em contraste, as contas de contrato apresentam características distintas das EOAs. Uma conta de
contrato abriga o código de um contrato inteligente, uma funcionalidade ausente em uma simples EOA.
Além disso, uma conta de contrato não possui uma chave privada. Em vez disso, é controlada pela
lógica incorporada no código do contrato inteligente \cite{6}.

As contas de contrato possuem endereços, semelhantes às EOAs, e também são capazes de enviar e
receber Ether. No entanto, quando uma transação é destinada a um endereço de contrato, ocorre a
execução desse contrato na \textit{Ethereum Virtual Machine} (EVM), utilizando a transação e os
dados da transação como entrada. Além do Ether, as transações podem conter dados que especificam
qual função específica do contrato deve ser ativada e quais parâmetros devem ser fornecidos a essa
função \cite{6}. Enquanto as EOAs permitem que os usuários tenham controle direto sobre seus fundos
e contratos, utilizando suas chaves privadas para autorizar transações, as contas de contrato
possibilitam a implementação de lógica programável e automatizada por meio do código do \textit
{smart contract}.

Com a compreensão das diferenças entre contas de propriedade externa (EOAs) e contas de contrato,
podemos agora explorar as aplicações descentralizadas (DApps) em maior detalhe. Essas aplicações
substituem a infraestrutura tradicional de back-end por smart contracts que operam em blockchains
como o Ethereum. Atualmente, o campo mais consolidado no contexto das DApps é o das finanças
descentralizadas. No entanto, a complexidade inerente ao uso das DApps tem sido um obstáculo para a
adoção em massa dessas aplicações.

Para enfrentar os desafios intrínsecos mencionados, são realizados esforços contínuos para
aprimorar e simplificar a experiência do usuário no ambiente do Ethereum. Como o Ethereum é uma
rede descentralizada e de código aberto, a comunidade constantemente formula propostas de melhoria
conhecidas como \textit{Ethereum Improvement Proposals} (EIPs). Uma proposta relevante para nossa
discussão é a EIP-4337 \cite{5}.

A EIP-4337, também conhecida como Account Abstraction (AA), introduz uma inovação radical no modelo
convencional de contas no Ethereum, sugerindo a utilização de smart contracts no lugar das
tradicionais EOAs. Essa proposta tem o potencial de abrir novos casos de uso na plataforma.

Ao proporcionar maior flexibilidade no processo de validação de transações, a Account Abstraction
(AA) desencadeia uma série de novas capacidades, destacando-se a possibilidade de autorização
personalizada sem a obrigatoriedade do uso de ECDSA, assim como nas EOAs. Isso permite a adaptação
da lógica de autorização de acordo com necessidades específicas \cite{1}. Esse caso de uso
específico que será o foco central deste artigo. Através da EIP-4337, podemos estabelecer regras de
validação que não dependem necessariamente da assinatura do proprietário da conta, o que representa
uma inovação para as blockchains. Até então, a autorização de transações era exclusivamente baseada
na atomicidade das assinaturas criptográficas, exigindo que a aprovação ocorresse instantaneamente,
sem a possibilidade de pré-aprovação.

\subsection{Account Abstraction}
Account Abstraction é um conceito que vem sendo explorado para aumentar a flexibilidade e
funcionalidade das contas na rede Ethereum.

Esse conceito sugere um desenvolvimento transformador na maneira como essas contas funcionam,
propondo que todas as contas na rede Ethereum tenham a potencialidade de operar como Contas de
Contrato. Isso implica que cada conta poderia abrigar sua própria lógica de operação, conduzindo a
um grau de personalização e funcionalidade sem precedentes. Por exemplo, uma conta poderia ser
programada para gerenciar transações de uma maneira específica ou para se defender contra certos
tipos de ataques.

\section{Desenvolvimento}\label{sec:desenvolvimento}
\subsection{Análise publicação da Visa}
\comment{text}{O texto abaixo precisa de referência}
Em sua recente contribuição para a expansão do campo de pagamentos automáticos programáveis em
carteiras auto-custodiadas via blockchain, a Visa apresentou revelações significativas numa
publicação técnica intitulada "Autopayments via Account Abstraction". As próximas linhas fornecem
uma síntese das conclusões primárias e propostas emergentes desta publicação.

Na sua abordagem, a Visa introduz um mecanismo inovador que simplifica a possibilidade do usuário
de executar autopagamentos, prescindindo do uso da chave privada associada à sua identidade. O
intuito subjacente é viabilizar autopagamentos para comerciantes sem a necessidade de expor a chave
privada do usuário a qualquer servidor de terceiros. Como alternativa, um contrato inteligente é
capacitado para processar um autopagamento em nome do usuário para o comerciante destinatário, sem
a exigência da chave privada do usuário.

O modelo concebido pela Visa concede ao contrato inteligente a autoridade de conduzir pagamentos
automáticos aos comerciantes associados ao usuário, estando sempre condicionado à ratificação deste
último. Esta ratificação pode ser obtida através do fornecimento de dados que o contrato
inteligente está autorizado a utilizar para realizar os autopagamentos em representação do usuário.
Em essência, o usuário pode pré-autorizar a transação, habilitando o contrato inteligente a
processar o pagamento em seu nome quando uma solicitação é encaminhada pelo comerciante. A Visa
sugere ainda que o usuário possa compilar uma lista de permissões, onde poderá pré-autorizar
transações com pagadores predeterminados, como comerciantes, outros usuários, entre outros.

Na monografia técnica mencionada, a Visa recorreu a um conceito recente e a uma das principais
propostas de desenvolvedores do Ethereum conhecida como Abstração de Conta para investigar a
implementação de contratos inteligentes que viabilizem pagamentos programáveis automáticos.
Propuseram uma inovadora solução para uma aplicação real de pagamentos automáticos, demonstrando
como estruturar um contrato inteligente para uma carteira autogerida capaz de retirar fundos
automaticamente, sem necessitar da participação ativa do usuário em cada instante para instruir e
realizar pagamentos numa blockchain.

\subsection{Abordagem da solução}

Demonstração de como os pagamentos automáticos podem ser implementados usando contratos
inteligentes de pagamento automático pré-aprovados escritos em Solidity.

\subsection{Implementação técnica}

Explicação dos passos técnicos necessários para configurar contas delegáveis em carteiras
auto-custodiadas. Visão sobre o fluxo de transação e interação entre o contrato inteligente de
pagamento automático e as carteiras controladas pelo usuário.

\subsection{Vantagens e Potenciais Aplicações}:
Discussão dos benefícios e vantagens oferecidos pela solução proposta para pagamentos automáticos.
Exploração de casos de uso potenciais além de pagamentos recorrentes, como serviços de recuperação de conta de terceiros e gestão de ativos.

\section{Conclusão}\label{sec:conclusao}
\comment{text}{Nem tente escrever conclusão ainda. Uma porque isso é só pra TCC II}
\remove{Reflexão sobre a importância dos pagamentos programáveis e o potencial para inovações futuras no
  espaço blockchain. Pesquisa Futura e Considerações Identificação de possíveis áreas para futuras
  pesquisas e desenvolvimento na automatização de pagamentos em carteiras de auto-gestão. Discussão
  sobre possíveis desafios e considerações a serem tratados na implementação de pagamentos
  automáticos em escala.}

\bibliographystyle{sbc}
\bibliography{tcc}

\end{document}

